\title{ {\bf Oil Sands Research }}
\author{
           \\
	  }
\date{Fall 2014}

\documentclass[12pt]{article}

\usepackage{amsmath}
\usepackage{amssymb}
\usepackage{amsthm}
\usepackage{color}
\usepackage{graphicx}
\usepackage{enumerate}
\setlength{\textwidth}{5.5in}
\setlength{\textheight}{8in}
\setlength{\voffset}{-1.1in}
\newcommand{\h}[1]{\colorbox{yellow}{#1}}
\newcommand{\problem}{\subsection*}
\newcommand{\R}{\mathbb{R}}
\newcommand{\F}{\mathbb{F}}
\newcommand{\Z}{\mathbb{Z}}
\newcommand{\Q}{\mathbb{Q}}
\newcommand{\C}{\mathbb{C}}
\newcommand{\Oo}{\mathcal{O}}

\begin{document}
\maketitle


%----------------------------------------------------------------------------------------
%	CONTENTS
%----------------------------------------------------------------------------------------

\tableofcontents

\newpage

\section{Introduction}

\begin{itemize}
\item Alberta Oil Sands Network
\item methods for oil sand development
\end{itemize}

Problem: Define $O$ the cost of a barrel of oil in the market, $E$ the energy required to develop a barrel of oil, $C$ the cost of producing 1 barrel of oil with a given method, $P$ the profit made by the oil company, and $X$ the amount of $CO_2$ released by producing that barrel. First, we want to model the following vector:

\begin{displaymath}
<O, E_m, C_m, P_m, X_m>
\end{displaymath}

In this paper we want to show that the vector $<O, E_{w} + E_{s}, C_{w} + C_{s}, P_{w+s}, X_{w+s}>$ yields better EROI while proving to be a good solutions towards a green future for Alberta. 

In other words, we want the following linear combination:

\begin{displaymath}
P = \underbrace{\alpha \times E_w \times C_w}_\text{EROI Wind Energy} + \underbrace{ \beta \times E_s \times C_s}_\text{EROI Solar Energy}
\end{displaymath}

In this paper we show that $\beta \times E_s \times C_s \approx 0$ does not prove to be a good EROI, hence $P \approx \alpha \times E_w \times C_w $\\


%%%%%%%%%%%%%%%%%%%%%%%%%%%%%%%%%%%%%%%%%%%%%%%%%
\section{Modelling}
%%%%%%%%%%%%%%%%%%%%%%%%%%%%%%%%%%%%%%%%%%%%%%%%%

What are we trying to model? What are my variables? What Algorithm do I need?
\subsection{Model Oil price of $\$/bbl$ in the next 50 years subject to price of last years using time series analysis}
\subsection{Model A: Current Alberta Oil Sands Network subject to $\$X/bbl$}
\subsection{Model B: Alternative Network subject to $\$/bbl$}
\subsection{Make the case that Model B better than Model A}

%%%%%%%%%%%%%%%%%%%%%%%%%%%%%%%%%%%%%%%%%%%%%%%%%
\section{Simulation}
%%%%%%%%%%%%%%%%%%%%%%%%%%%%%%%%%%%%%%%%%%%%%%%%%
What are we trying to simulate? What are my simulation environments? What algorithm do I need?

\subsection{Model A: Simulation of Current Alberta Oil Sands Network subject to $\$X/bbl$}
\subsection{Model B: Simulation of Alternative Network subject to $\$/bbl$}


%%%%%%%%%%%%%%%%%%%%%%%%%%%%%%%%%%%%%%%%%%%%%%%%%
\section{Optimization}
%%%%%%%%%%%%%%%%%%%%%%555555%%%%%%%%%%%%%%%%%%%%%%%%
What are we trying to optimize? What are my constrains? What algorithms do I need?

\subsection{Optimization of Model B}

\section{Conclusion}
Make the case for Model B better than Model A. 

\newpage

\section{Notes with SIMON}
\subsection{Table of contents}
Applied math in industry
1) modelisation
  - we have data, we don't know if this data is useful or not, we need to show that we
need to show in our model for achiving results, and data to feed that model. 
2) simulation
we have a lot of data that depend on real models. identify scenarios. and simulate to 
show that this configuration is better than the conventional configuration 
3) optimization 
what are the most important factors. which elements take into account. most important
variables. 


\subsection{Intro}
This is not a mere Minimizing/Maximizing optimization problem in which we are trying to minimize the CO2 amount while maximizing profits as simply as that, but it is rather
a modern complex optimization problem combining solutions from Discrete, Combinatorial,
Stochastic, and Multi-Objetive algorithms. 

We discuss the combination of multiple methods to find a better configuration than
the conventional configuration, meta-huristic algorithms. 

Goal: go search alberta funding. 
 

\subsection{Discrete Optimization}
the network is a discrete network, the choice of putting a turbine or not is also
discrete (yes or not). 

how many do we place? a bit less a bit more? amortize it.

\subsection{Cominatorial Optimization}
this a combinatorial optimzation problem. there're many heuristics exist for this. 
"recherche tabu", "algorithme genetique", in concrete it is a problem with complex
structural optimization. 
For the whole network of turbines, that's a combinatorial problem. meta-huristics there


\subsection{Stochastic Optimization}
When talking about the functioning of the turbine, there's data problems.
there we can talk about machine learning because the turbine needs to learn when to shut down when there is no wind. this is stochastic optimization problem. We have a real 
system where the wind is random variable, we have the choice of keeping the turbine on or to turn it off, and if we don't do this at the right time, turbine could be non-functioning anymore. 

the stochastic problem is at the center of one wind turbine. when to turn it on/off.

\subsection{Multi-objective optimization}
we want to maximize something and minimize something. Is there a direct relationship?
Maybe be not. This what we call multi-objective optimization (profit, co2). When we have conflicting variables, there are artificial trade-offs. 

CO2 evaluation methods are highly political. Need to play with this 2 tables (profit, co2)

\subsection{Conclusion}
We showed that this configuration is better than the conventional configuration provided the simulation scenarios. 



\end{document}
