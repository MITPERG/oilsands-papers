\title{ {\bf Oil Sands Research }}
\author{
           Paper\#2\\
	  }
\date{Spring 2015}

\documentclass[12pt]{article}

\usepackage{amsmath}
\usepackage{amssymb}
\usepackage{amsthm}
\usepackage{color}
\usepackage{graphicx}
\usepackage{enumerate}
\setlength{\textwidth}{5.5in}
\setlength{\textheight}{8in}
\setlength{\voffset}{-1.1in}
\newcommand{\h}[1]{\colorbox{yellow}{#1}}
\newcommand{\problem}{\subsection*}

\begin{document}

\maketitle
\tableofcontents
\newpage

%%%%%%%%%%%%%%%%%%%%%%%%%%%%%%%%%%%%%%%%%%%%%%%%%
\section{Introduction}
%%%%%%%%%%%%%%%%%%%%%%%%%%%%%%%%%%%%%%%%%%%%%%%%%

\subsection{Mise-en-Contexte}
The oil sands industry has experienced two challenging events in the past months. First of all, the continous drop of oil price and the rejection of Keystone XL by the Obama administration. As a consequences, oil companies started to pull the plug on Alberta expansions and cutting down the expenses under a break-even threshold analyst say is needed to justify a bran new oil sand expansion.. In addition, companies are trimming spending plans and reducing workforces in response to this. Another twist to the story is that investors are diverted to U.S shale oil plays.  \\

  Mr. Laut said oil sand producers were making three times the profit in 2004 when a barrel of oil cost about \$40 U.S than it did when the price hit close to \$100 in 2013. What happens is the rising cost of supplier. Two important points follow:
  
  \begin{enumerate}
  \item producers have to be more efficient and develop new technologies that lower production costs
  \item Cut costs and eliminate inefficiencies
  \end{enumerate}

Shell's focus: ``Our current focus is on making our heavy oil business as economically and environmentally competitive as possible''

\subsection{About}
In this paper we study the Oil Sands EROI in Alberta, Canada as a multi-objective optimization problem in order to show that if we use renewables to produce oil sands then not only we can achieve a CO2-free industry, but also a more profitable industry. 

In this paper, we focus on: 
\begin{enumerate}
\item {\bf Modelling} \\
We have data, we don't know if this data is useful or not, we need to show that we
need to show in our model for achiving results, and data to feed that model. 
\item {\bf Simulating} \\
We have a lot of data that depend on real models. identify scenarios. and simulate to 
show that this configuration is better than the conventional configuration 
\item {\bf Optimazing} \\
What are the most important factors. which elements take into account. most important
variables. \\
\end{enumerate}

{\em Process Insight:} Oil sands upstream processing
{\em Asset Insight:} Equipment Realiability and Maintenance

The Oil Sands industry is not a simple and direct Minimizing/Maximizing optimization problem in which we are trying to minimize the $CO_2$ amount while maximizing profits as simply as that, but it is rather a modern complex optimization problem combining solutions from Discrete, Combinatorial, Stochastic, and Multi-Objetive algorithms. \\

In this paper, we discuss the combination of multiple methods to find a better configuration than the conventional configuration through meta-huristic algorithms. \\

DATA ->Build Model  by training extact patterns -> Geologic Block Data by Applying rules -> Visualize One Body Map or Visualize One Processability Map

\subsection{Current Supply \& Demand}

Currently, there is more supply than demand. In this paper, we are interested in studying supply vs deman from supply ``need'' point of view. The price of oil has been fluctuating a lot in the past months, yet people basic needs have been evolving more slowly. The price of gas is not considerably lower and we want to address the issue of  supply from a concrete need of the population. We will talk about elastic and inelastic demand, this will give an interesting perspective to the paper. \\

The highest price for a barrel was above \$100 in 2013 [ref]. However, oil sands idustries were making their highest profits back in 2007, when the price of a barrel of oil was \$54 [ref]. Therefore, industries were making four times as much profit was they were making it today. What changed? As the price of a barrel of oil went up, so did all the supplier costs and third party services. Now, the price is down, but the cost of suppliers and other vendors is not down, but they remain the same. That's what the main issue is right now. What we want to do is find ways to reduce supplier costs. They have not reached a technological peak so that the supply of these suppliers is affordable.\\

TO ADD: the situation regarding Keystone XL

\subsection{Introduction to Multi-objective Optimization Problem (MOP)}

A multi-objective optimization problem consists in a set of objectives (discrete variables) to which we associate a cost function we want to optimize. Th set of optimized solutions is the pareto optimal set. \\

We address the Oil Sands EROI problem from the following perpective: \\

{\bf Definition:} Define $B$ the cost of a barrel of oil in the market, $E$ the energy required to develop a barrel of oil, $C$ the cost of producing 1 barrel of oil with a given method, $P$ the profit made by the oil company exploiting the bitumen, and $X$ the amount of $CO_2$ released by producing that barrel. We want to model the following cost vector $v$ as a multi-objective optimization problem:

\begin{displaymath}
v = \langle B, E, C_B, P, X \rangle
\end{displaymath}

We want to study this cost vector $v$ based on different scenarios (or weights on different variables)and different methods of producing a barrel of oil using renewables. Depending on our decisions, we will weight some variables more than others. This vector $v$ will serve more and an indicator of performance for different EROI models. As an analogy, consider how different organizations provide university rankings based on different criteria (number of PhDs, research grants, Nobel Prize winners, etc), this vector will behave in a similar basis.  Changing variable weights will return different solutions for different scenarios. \\

This EROI analysis We can bring this discussion to the table. If we agree to weight into a certain variable more than another one, we the domocratically judge what model to focus on based on which multi-objective function. This will clarify the choices of the model we will expand our work on. It will come down to how we weight these variables in question that will define the best working model for the problem. Ultimately, there is no mathematics that does this, it will depend on human decisions. \\

In this paper, we analyze the EROI problem through different situations to find which variables do we have control over, which variables we do not have control over (paramaters and coefficients), and which variables we want to optimize. In this paper, we want to show that the vector 

\begin{displaymath}
v_{wind+solar}= \langle B, E_{wind} + E_{solar}, C_{wind} + C_{solar}, P_{wind + solar}, X_{wind+solar}\rangle
\end{displaymath}
 yields better EROI while proving to be a good solutions towards a green future for Alberta. \\

In other words, we want the following linear combination:

\begin{displaymath}
P_{wind+solar} = \underbrace{\alpha \times E_w \times C_w}_\text{EROI Wind Energy} + \underbrace{ \beta \times E_s \times C_s}_\text{EROI Solar Energy}
\end{displaymath}

In this paper we show that $\beta \times E_s \times C_s \approx 0$ does not prove to be a good EROI, hence $P \approx \alpha \times E_w \times C_w $\\

Notes to add:
\begin{itemize}
\item Specify what we mean by $P$. If different companies have different profits then we talk about another vector of prices, different companies have different prices associated with the landscape they work on.
\item Specify what me mean by $CO_2$.  $CO_2$ produced by oil sands companies released in the atmosphere 
\item Alberta Oil Sands Network
\item methods for oil sand development
\end{itemize}

\subsection{Method}
Simulation: simulate on different objectives and see. Conclude different propositions as a result. \\

First Approach: Brute-Force. Find out how much time it takes for the computer to find a solution. If we need to change a specific variable (say the price of oil again), then it will take again a certain amount of time to solve. \\

We will first try to cut down the time by a more global approach: 1) Branch and Bound 2) Tabu Search . We will test how this evolves and how different scenarios produce different solutions through these methods.  \\

We will try to test in parallel all Meta-Heuristics: 1) VEGA 2) TAPAS. \\

The final step is to take approx solutions to approx problems. \\

%%%%%%%%%%%%%%%%%%%%%%%%%%%%%%%%%%%%%%%%%%%%%%%%%
\section{Modelling}
%%%%%%%%%%%%%%%%%%%%%%%%%%%%%%%%%%%%%%%%%%%%%%%%%

What are we trying to model? What are my variables? What Algorithms do I need?

Notes:
\begin{itemize}
\item Model Oil price of $\$/bbl$ in the next 50 years subject to price of last years using time series analysis
\item Model A: Current Alberta Oil Sands Network subject to $\$X/bbl$
\item Model B: Alternative Network subject to $\$/bbl$
\item Make the case that Model B better than Model A
\item Path Linking Method. We look at the neighborhood of the solution space. 2-objective functions gives birth to 2 grids. Cube space (a,b,c) a=Wind, b=Solar, c=Mining. Different configurations can represent something interesting things. Our topology map will be encoded based on our contrains. Then we need to establish rules, for instance, (1,1,0) can be next to (1,0,1), but not next to (0,1,1) we encode the solution structure. Given X, this is the neighborhood.  
\end{itemize}

%%%%%%%%%%%%%%%%%%%%%%%%%%%%%%%%%%%%%%%%%%%%%%%%%
\section{Simulation}
%%%%%%%%%%%%%%%%%%%%%%%%%%%%%%%%%%%%%%%%%%%%%%%%%
Notes:
\begin{itemize}
\item The first approach will be by exahustive enumeration (brute force). 
\item Second approach is a branch and bound approach where we are interested in cutting down the number of cases to consider in a decision tree. To explore the heuristics, we take this variable in the grid. We try with another variable. Some discrete variables can be used as continuous. 
\item We try to find a pareto optimal. But first, we go by scenario-approach. We look for pareto local solutions. We look at different pareto local solutions based on different scenarios. Looking at the model, we might introduce new constrains that we never though before. This process allows to illicit their contrains or needs depending on the choices we make. Then, we can possible arrive at a pareto optimal solution. 
\end{itemize}

\subsection{Techniques}
\begin{enumerate}
\item Using generic algorithms (meta-heuristics)
\item Non-scalar and non-pareto approch: VEGA (Vector Evaluated Genetic Algorithm)
\item Target Amining Pareto Search (TAPAS)
\item Path Relinking Algorithm: Neighborhood Exploration. Add graph.  
\item Domination Dependent lifetime: maximal lifetime is assigned to each wind turbine. The lifetime is shortened if the solution dominates a major part of the present non-dominated solutions. This limits the impact of the solution. 
\end{enumerate}

What are we trying to simulate? What are my simulation environments? What algorithm do I need? \\

Data set analysis. 


%%%%%%%%%%%%%%%%%%%%%%%%%%%%%%%%%%%%%%%%%%%%%%%%%%
\section{Optimization}
%%%%%%%%%%%%%%%%%%%%%%555555%%%%%%%%%%%%%%%%%%%%%%%%
In the pursuit of finding a pareto optinal solution, we will study different pareto optimums which represent different choices because our cost vector  $v$ does not have more information. We give the pareto optimum solutions scenarios, we cannot distinguish them if we don't weight in some variables more than others, so we are left with a choice. Then, based on that choice, we will show what we have found. 

What are we trying to optimize? What are my constrains? What algorithms do I need?

\subsection{Discrete Optimization}
the network is a discrete network, the choice of putting a turbine or not is also
discrete (yes or not). 

how many do we place? a bit less a bit more? amortize it.

\subsection{Cominatorial Optimization}
this a combinatorial optimzation problem. there're many heuristics exist for this. 
"recherche tabu", "algorithme genetique", in concrete it is a problem with complex
structural optimization. 
For the whole network of turbines, that's a combinatorial problem. meta-huristics there


\subsection{Stochastic Optimization}
When talking about the functioning of the turbine, there's data problems.
there we can talk about machine learning because the turbine needs to learn when to shut down when there is no wind. this is stochastic optimization problem. We have a real 
system where the wind is random variable, we have the choice of keeping the turbine on or to turn it off, and if we don't do this at the right time, turbine could be non-functioning anymore. 

the stochastic problem is at the center of one wind turbine. when to turn it on/off.

\subsection{Multi-objective optimization}
we want to maximize something and minimize something. Is there a direct relationship?
Maybe be not. This what we call multi-objective optimization (profit, co2). When we have conflicting variables, there are artificial trade-offs. 

CO2 evaluation methods are highly political. Need to play with this 2 tables (profit, co2)

\subsection{Performance Assessment}
\begin{itemize}
\item How to measure the quality of the solution?? 
\item How much time does it take to generate the solution set??
\item What we find in this paper is that we don't find a single solution, but rather a set of solutions based on trade-offs
\end{itemize}


%%%%%%%%%%%%%%%%%%%%%%%%%%%%%%%%%%%%%%%%%%%%%%%%%
\section{Results}
%%%%%%%%%%%%%%%%%%%%%%%%%%%%%%%%%%%%
\subsection{CO2-Free solution long term}
\begin{itemize}
\item More progress towards returning the landscape back
\end{itemize}


\subsection{A profitable business for oil companies}
\begin{itemize}
\item More \$ for refineries and pipelines
\item (SAMPLE) The technology resulted in an additional 1,600 barrels of bitumen per day being recovered from just one vessel. The figure means that 50\% less marerial ends up in tailing ponds, and at oil prices \$50 per barrel translates into an additional 30 million in annual revenues for Suncor. 
\end{itemize}

%%%%%%%%%%%%%%%%%%%%%%%%%%%%%%%%%%%%%%%%%%%%%%%%%
\section{Conclusion}
%%%%%%%%%%%%%%%%%%%%%%%%%%%%%%%%%%%%%%%%%%%%%%%%%
The big picture: what do we propose? and to whom do we propose to? Then a several things to consider. \\

We proposed the problem in this way. We showed what can we do about it. We showed the model created. We showed different action scenarios which implied this and that result. If there is something in these results that is of interest to the energy sector we then plan to expand on that specific branch and show what has to be done for that to happen.\\  

We showed that this configuration is better than the conventional configuration provided the simulation scenarios. \\

The difference with this problem and other problem in industry is that at the end of the day we have a choice. We can go further and propose things to politicians and oil companies. We can propose things and see how the problem is approached. \\

END USER ACTION FOCUSING -> High Order Complex relationship -> Transparency (Provide an easy way to understand outcomes  instead of black box approaches) -> Multiple Data Sources (handling Unstructured data) 

In the end, we find a better balance between profit margins and public safety . 

\section{References}
To Add. 

\end{document}
